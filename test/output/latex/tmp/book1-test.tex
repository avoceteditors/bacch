\documentclass[openright,pdflatex,10pt]{book}

\usepackage{setspace}
\usepackage[framemethod=default]{mdframed}
\usepackage{lettrine}
\usepackage{textcase}
\usepackage{xcolor}
\usepackage{titletoc}
\usepackage{listings}
\usepackage[b5paper]{geometry}
\usepackage{framed}
\usepackage[utf8]{inputenc}
\usepackage[explicit,noindentafter]{titlesec}
\usepackage[english]{babel}
\usepackage{pifont}
\usepackage{scrextend}
\usepackage[tracking]{microtype}
\usepackage{fancyhdr}
\usepackage{bookmark}


\frenchspacing
\definecolor{shadecolor}{rgb}{1,0,0}
\mdfsetup{backgroundcolor=gray}
\newmdenv{code}
\begin{document}


\chapter*{Chapter 1: Abstract Tests}


The \emph{main} goal of the abstract test is to render the block in the containing text of links.  For instance, in MySQL documentation if I hover over a link for SELECT statements, I want to see a short description of what SELECT does.


\begin{itemize}
\item
      \textbf{Current Link Test}: Tests XLink to current file: chap0101.

\item
      \textbf{External Link Test}: Tests XLink to external file: chap0101-sect1.

\item
      \textbf{Internal Link Test}: Tests XLink to a section in the current file: chap0101-sect2.

\item
      \textbf{External Link Test}: Tests XLink to external resource: http://en.wikipedia.org
\item
      \textbf{Custom Title Link Test}: Tests XLink with custom title: Custom Title
\item
      \textbf{Link Format Test}: Tests XLink with custom formatting: chap0101-sect3

\end{itemize}


In the event that all tests pass, you should see links running through the above items.  With the exception of the Custom Title test, failing tests should appear as blank spaces after the colons, or some other unexpected results.



Resource titles change over time as projects develop.  Rather than requiring the writer to rewrite link text every time there's a change in titles, Bacch supports custom rendering of title text. <link xlink:href="resource"/> retrieves the title from the given resource along with the abstract text.  Additionally, setting the bacch:format attribute on the resource block allows you to apply formatting variables to the output.

\section*{Section 1: Basic Text}


For this section I want to see whether the difference between chapters and sections affects how abstract text is handled.

\section*{Section 2}
\section*{Section 3}


For this test, I want to see if I can render formatting variables into link output.  For instance, in the case of MySQL documentation, when I link to pages like SELECT I want the output to render with a monospace font in links.

\chapter*{Chapter 2: Formatting Tests}


The purpose of this chapter is to test how Bacch renders inline formatting options.  In the list below, blank spaces after the colon indicate failed tests:


\begin{itemize}
\item

Bold Test: \textbf{bold}
\item

Emphasis Test: \emph{italic}
\item

Code Test: code

\end{itemize}
\chapter*{Chapter 3: Block Tests}


The purpose of this chapter is to test various block-level rendering, such as code blocks and blockquotes.  Content that renders unexpectedly or blank in the section below indicate test failures.


\begin{itemize}
\item

Basic Code Block


      \begin{code}
content for basic code block, no formatting.
\end{code}

\item

Code block with syntax highlighting.


      \begin{code}
SELECT * FROM tab
\end{code}

\item

Code Block with input and output, using the standard <prompt> element.


      \begin{code}$
cat test.txt
stuff
stuff
morestuff\end{code}

\item

Bacch supports variable prompts, allowing you to define PS1 and PS1 values and a system level.


      \begin{code}
SELECT FROM MyClass
WHERE name = "id"
\end{code}

\item

Rendering tables such as MySQL output can prove a hassle.  Design goal for Bacch is to implement a handler to adjust table variables in <computeroutput>.


      \begin{code}
SELECT * FROM my_table;

+--+--+
| Variable_name | Value |
+--+--+
| wsrep_on | ON |
+---+---+
\end{code}

\item

Test for replaceable variables in <programlisting> blocks.


      \begin{code}
SELECT * FROM table;
\end{code}


\end{itemize}



\end{document}
